\section{Introduction}

\begin{frame}{Introduction}
    Table of contents
    \begin{itemize}
        \item The Huffman Coding Algorithm
        \item Serial Implementation
        \item Parallel Implementation
        \item Performance Evaluation and Benchmarks
        \item Q\&A
    \end{itemize}
\end{frame}

\begin{frame}{The Huffman Coding Algorithm}
    Huffman Coding is a \textbf{lossless data compression} algorithm  designed to find a more convenient bit representation to store data through variable-length sequences of bits defined as \emph{alphabet}.
    \begin{table}[]
        \centering
        \begin{tabular}{lll}
            \toprule
            ASCII character & Byte encoding & Huffman encoding \\
            \midrule
             a &  01100001 & 00 \\
             b &  01100010 & 010 \\
             c &  01100011 & 011 \\
             d &  01100100 & 10 \\
             e &  01100100 & 11 \\
             \bottomrule
        \end{tabular}
        \caption{Example of Huffman alphabet for 5 letters}
        \label{tab:my_label}
    \end{table}
\end{frame}

\begin{frame}{Frequency-based Encoding}
    The Huffman code for each character is decided by the occurrences of that character in the text using a \textbf{greedy} procedure:
    \begin{itemize}
        \item more frequent -> less bits
        \item less frequent -> more bits
    \end{itemize}
\end{frame}

\begin{frame}{Optimal Prefix Code}
    Even if the the Huffman algorithm is based on a greedy approach, it is able to generate an \textbf{optimal prefix code} in space efficiency.
\end{frame}
