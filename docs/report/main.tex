\documentclass[conference]{IEEEtran}
\IEEEoverridecommandlockouts
% The preceding line is only needed to identify funding in the first footnote. If that is unneeded, please comment it out.
\usepackage{cite}
\usepackage{amsmath,amssymb,amsfonts}
\usepackage{algorithmic}
\usepackage[hidelinks]{hyperref}
\usepackage[noabbrev,capitalize]{cleveref}
\usepackage{array, booktabs, makecell}
\usepackage{graphicx}
\usepackage{textcomp}
\usepackage{xcolor}
\usepackage{caption}
\usepackage{subcaption}
\newcommand\norm[1]{\left\lVert#1\right\rVert}
\def\BibTeX{{\rm B\kern-.05em{\sc i\kern-.025em b}\kern-.08em
    T\kern-.1667em\lower.7ex\hbox{E}\kern-.125emX}}

\graphicspath{../}
\hypersetup{
    colorlinks=true,
    linkcolor=black,
    filecolor=black,      
    citecolor=black,
    urlcolor=blue,
    pdfpagemode=FullScreen,
    }

\begin{document}

\title{Parallel Huffman Coding}

\author{\IEEEauthorblockN{Francesco Bozzo}
    \IEEEauthorblockA{\textit{DISI, University of Trento} \\
        Trento, Italy\\
        francesco.bozzo@studenti.unitn.it\\
        229312}
    \and
    \IEEEauthorblockN{Michele Yin}
    \IEEEauthorblockA{\textit{DISI, University of Trento} \\
        Trento, Italy\\
        michele.yin@studenti.unitn.it\\
        229359}
}

\maketitle

% add page number
\thispagestyle{plain}
\pagestyle{plain}

\begin{abstract}
    This report aims to explain the design of a parallel encoding and decoding Huffman algorithm. The application is developed with the C99 programming language, using MPI for multiprocessing and OpenMP for multithreading to scale horizontally with increasing hardware resources. \cite{mapnet}
\end{abstract}

\begin{IEEEkeywords}
    Huffman, MPI, OpenMP, High Performance Computing
\end{IEEEkeywords}

\section{Introduction}
The \emph{Huffman algorithm} is designed to find a more convenient bit representation, to store data through lossless compression. Instead of considering groups of eight bits to encode data, the Huffman algorithm uses variable length sequences of bits defined as \emph{alphabet}.

The objective of our work is to build a scalable Huffman encoder and decoder that are able to scale according to the provided hardware resources. Moreover, the application should handle both single files and nested folders properly.

Even if nowadays the Huffman algorithm is mainly used for teaching purposes, its prefix mechanism is still part of many notable standards, such as Deflate (PKZIP's algorithm), JPEG, and MP3 compression algorithms. For this reason, there are no state-of-the-art online-available implementations to compare our tool with. In fact, differently from our work, many of the tools we found are not built considering low-level optimizations (such as to use in-memory buffers to improve I/O timings) or non-essential features (such as folders and big files handling).

\section{Conclusions}
To conclude.


\bibliography{bibliography.bib}{}
\bibliographystyle{IEEEtran}

\end{document}
