\section{Serial Huffman implementation}
To be able to implement the Huffman encoding and decoding algorithm, several things have been taken into account.
\begin{itemize}
    \item The byte frequencies computed during the encoding process are saved in the compressed file. In this way, the decoding procedure can easily rebuild the Huffman tree.
    \item Both the encoding and decoding procedures make use of buffers to improve I/O performance: the streams of bytes are first written inside the buffer and then saved on the disk.
    \item The encoding procedure works with chunks of \(4096\) bytes. This ensures that the tool can handle even large files when dealing with bit buffers. The decoding procedure deals with chunks of maximum size of \(4096*32\) bytes, since the maximum length of a Huffman tree traverse from root to leaf is of 256 bits. As explained later, dealing with chunks makes handy the parallel implementation of the encoding and decoding algorithms.
    \item In the encoded file, the chunk offsets are saved at the end of the file: this ensures that the compressed stream of prefixed sequences of bits is still meaningful.
\end{itemize}
